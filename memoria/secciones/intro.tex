\chapter{Introducción}
\label{ch:intro}
En este apartado se introduce la problemática de la gestión de operaciones en aeropuertos. Se presentarán los objetivos principales de este trabajo para estudiar el flujo de aeronaves y el rendimiento de los recursos críticos del sistema. Finalmente, se expondrá la estructura general que seguirá el proyecto y la organización del documento.
\section{Motivación del proyecto}
El presente Trabajo de Fin de Grado surge como respuesta al crecimiento sostenido del sector aéreo, una tendencia que está llevando al límite la capacidad operativa de las infraestructuras aeroportuarias actuales. Esta tendencia crea cuellos de botella que terminan convirtiéndose en retrasos y malas experiencias para los viajeros.
\\
\\
Por ello, es fundamental poder anticiparse. Desarrollar un simulador es la estrategia más inteligente, ya que nos permite crear escenarios complejos y medir la eficiencia del aeropuerto de manera segura y barata.

\section{Definición del problema}
En los últimos años, la creciente complejidad de las operaciones aeroportuarias y el aumento del tráfico aéreo han hecho que la gestión de los aeropuertos se convierta en un desafío tanto estratégico como operativo. La planificación de flujos de pasajeros, la coordinación de operaciones en pista y la optimización de los recursos requieren herramientas que permitan analizar diferentes escenarios posibles e intentar anticipar posibles fallos y cuellos de botella sin que esto afecte a la operación real.
\\
\\
En este contexto, la simulación de eventos discretos se ha consolidado como una metodología efectiva para evaluar procesos e identificar problemas potenciales, ya que permite reproducir de manera controlada el comportamiento de sistemas complejos en función del tiempo y de la interacción entre múltiples recursos.
\\
\\
El presente Trabajo de Fin de Grado se centra en el estudio y modelización del flujo de aeronaves, con especial atención a los procesos de aterrizaje, estacionamiento y despegue, así como la influencia de los servicios auxiliares en los tiempos operativos. El sistema se ha implementado mediante técnicas de análisis de datos y metodologías estadísticas. El sistema de simulación produce información estructurada sobre el flujo de aeronaves y las operaciones del aeropuerto, incorporando variaciones debidas a temporadas de vacaciones, condiciones climáticas y otros eventos que pueden causar retrasos o cambios en la programación de vuelos.
\\
\\
En resumen, este Trabajo de Fin de Grado busca ofrecer una aproximación práctica al análisis de operaciones aeroportuarias mediante la simulación, aportando una herramienta que contribuya a la gestión eficiente de operaciones en entornos de alta complejidad.
\section{Objetivos}
El objetivo general de este trabajo es desarrollar un modelo de simulación de eventos basado en tiempo discreto que reproduzca el flujo de aeronaves en un aeropuerto, incluyendo las fases de llegada, aterrizaje, estacionamiento y salida, así como los procesos auxiliares asociados y factores externos.
\\
\\
De este objetivo general se han definido una serie de metas que son las siguientes:
\begin{itemize}
    \item \textbf{Definir y modelar el sistema aeroportuario:} identificando los componentes físicos, los flujos de aeronaves y los factores empíricos que impactan la operación.
    \item \textbf{Formular la metodología matemática:} basándose en la \textit{Teoría de Colas} y en procesos estocásticos para modelar el flujo operacional aeroportuario.
    \item \textbf{Implementar y validar un prototipo funcional:}  que gestione correctamente la concurrencia de eventos y el uso de recursos compartidos.
    \item \textbf{Analizar e interpretar los resultados obtenidos:} evaluando métricas clave, como los tiempos de espera y los cuellos de botella, para proponer mejoras operativas.
\end{itemize}

\section{Estructura del documento}

En esta sección se redacta un breve listado con las secciones de interés del trabajo. Está compuesto por capítulos y anexos.
\begin{itemize}
    \item \textbf{Capítulo 2}. Está dedicado al estudio del arte, donde se analizan los principales conceptos teóricos, trabajos previos y metodologías existentes.
    \item \textbf{Capítulo 3}. Describe el modelo empírico descriptivo del proyecto, detallando los componentes que definen el funcionamiento de un aeropuerto en el mundo real.
    \item \textbf{Capítulo 4}. Detalla el desarrollo e implementación del primer prototipo.
    \item \textbf{Capítulo 5}. Detalla el desarrollo e implementación del segundo prototipo.
    \item \textbf{Capítulo 6}. Presenta los resultados obtenidos a partir de la simulación y recoge las conclusiones obtenidas tras el desarrollo del proyecto.
    \item \textbf{Bibliografía y Anexo}. Presenta la bibliografía, los anexos de código y resultados, y las referencias empleadas a lo largo del trabajo.
\end{itemize}

