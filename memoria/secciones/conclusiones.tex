\chapter{Resultados y Conclusiones}
Una vez implementados y validados ambos prototipos, es necesario analizar los datos obtenidos. En este capítulo se comparan ambos prototipos con el objetivo de evaluar cuál de ellos ofrece un mayor grado de realismo y una representación fiel del flujo de operaciones y sistemas auxiliares. Por último, se presenta una síntesis de los resultados más relevantes del estudio y se proponen algunas ideas para mejorar y ampliar el modelo en trabajos futuros.
\section{Análisis Comparativo de los Prototipos}
El objetivo de esta sección es contrastar los resultados obtenidos en ambos prototipos para identificar las principales diferencias y los cambios más relevantes entre ellos. Mientras que el primer prototipo sirvió como una base inicial, pero muy completa, del modelo, mientras que el segundo se desarrolló con el objetivo de incoporar un mayor nivel de realismo con nuevas lógicas y funcionalidades.
\\
\\
A continuación, se presenta una comparativa mediante tablas en la que se analizan las diferencias entre los escenarios de cada prototipo. En ellas se comparan los aspectos clave de ambos modelos, con el objetivo de identificar en qué puntos difieren y cómo evolucionan entre una versión y otra.
\\
\\
Para comenzar se va a realizar una comparativa del Escenario Base de cada prototipo:
\begin{table}[H]
    \centering
    \renewcommand{\arraystretch}{1.3}
    \begin{tabular}{|l|c|c|} 
        \hline
        Criterio & Prototipo 1 & Prototipo 2\\
        \hline
        Comportamiento General & Lineal y Determinista & Estocástico y Variable\\
        Sensibilidad a la Carga & Baja & Alta\\
        Saturación & Inexistente & Evidente\\
        Tiempo Medio de Ciclo & \textasciitilde81 minutos & \textasciitilde115 minutos\\
        Sesgo & Optimista & Conservador\\
        Variabilidad del IC & Media ($\pm3$) & Alta ($\pm5$)\\
        Presencia de Valores Atípicos & Nula & Evidente\\
        Consistencia del Flujo & Alta & Alta\\
        \hline
    \end{tabular}
    \caption{Comparativa Escenario Base}
    \label{tab:Comparativa Escenario Base}
\end{table}
Tal y como se observa en la Tabla \ref{tab:Comparativa Escenario Base}, el paso del primer al segundo prototipo supone una clara mejora del modelo, ya que se pasa de un enfoque lineal y fijo a uno estocástico. El cambio más notable se observa en el Tiempo Medio de Ciclo, que aumenta de 81 minutos en el primer prototipo a aproximadamente 116 minutos en el segundo prototipo. Es importante destacar que este incremento del 42\% no se debe a un peor funcionamiento, sino a que el primer modelo era demasiado optimista.
\\
\\
Mientras que en el Prototipo 1 la situación era ideal, ya que no había servicios auxiliares, cierres de pista, ni tiempos de espera entre aterrizajes, el Prototipo 2 tiene en cuenta más condiciones realistas del entorno aeroportuario, como la variabilidad en los tiempos de servicio y la llegada de vuelos en estado de emergencia. Además, el análisis de valores atípicos y del intervalo de confianza, que revela una diferencia importante entre ambos modelos. Mientras que el primer modelo presentaba un intervalo estrecho ($\pm3$), lo que no era tan realista, el segundo modelo muestra un intervalo más amplio ($\pm5$), De este modo, el sistema refleja de forma más honesta los problemas operativos y confirma una mayor robustez del modelo.
\begin{table}[h!]
    \centering
    \renewcommand{\arraystretch}{1.3}
    \begin{tabular}{|l|c|c|} 
        \hline
        Criterio & Prototipo 1 & Prototipo 2\\
        \hline
        Comportamiento General & Caótico e Inestable & Robusto y Predecible\\
        Sensibilidad a la Carga & Crítica & Alta\\
        Saturación & Colapso & Gestionada\\
        Tiempo Medio de Ciclo & \textasciitilde200 minutos & \textasciitilde169 minutos\\
        Sesgo & Ineficiente & Resiliente\\
        Variabilidad del IC & Muy alta ($\pm7$) & Media ($\pm3.5$)\\
        Presencia de Valores Atípicos & Descontrolada & Controlada\\
        Consistencia del Flujo & Alta & Alta\\
        \hline
    \end{tabular}
    \caption{Comparativa Escenario de Estrés}
    \label{tab:Comparativa Escenario de Estrés}
\end{table}
\\
Al poner ambos prototipos a prueba en un escenario de estrés, los resultados de la Tabla \ref{tab:Comparativa Escenario de Estrés} son bastante claros. El Prototipo 1, que en situaciones normales parecía funcionar bastante bien, en este escenario se viene abajo, ya que el tiempo medio se dispara a 200 minutos y, además, la incertidumbre aumenta mucho ($\pm7$), lo que indica que el modelo deja de ser estable y se vuelve difícil de predecir.
\\
\\
En cambio, el Prototipo 2 demuestra que está mejor preparado para este tipo de situaciones. Consigue un tiempo medio considerablemente menor (169 minutos) y, además, mantiene la variabilidad más controlada ($\pm3.5$). Por tanto, esto confirma que, cuando el aeropuerto se satura, el segundo modelo es capaz de gestionar mejor las colas y organizar el tráfico.
\begin{table}[h!]
    \centering
    \renewcommand{\arraystretch}{1.3}
    \begin{tabular}{|l|c|c|} 
        \hline
        Criterio & Prototipo 1 & Prototipo 2\\
        \hline
        Comportamiento General & Inestable y Volátil & Resiliente y Estable\\
        Sensibilidad a la Carga & Muy Alta & Alta\\
        Saturación & Degradación Aleatoria & Degradación Lineal\\
        Tiempo Medio de Ciclo & \textasciitilde171 minutos & \textasciitilde160 minutos\\
        Sesgo & Impreciso & Preciso\\
        Variabilidad del IC & Muy alta ($\pm7$) & Baja ($\pm2.5$)\\
        Presencia de Valores Atípicos & Frecuente & Acotada \\
        Consistencia del Flujo & Alta & Alta\\
        \hline
    \end{tabular}
    \caption{Comparativa Escenario Adverso}
    \label{tab:Comparativa Escenario Adverso}
\end{table}
\\
Por último, la Tabla \ref{tab:Comparativa Escenario Adverso} muestra claramente el comportamiento de ambos modelos en condiciones desfavorables. Se observa que el Prototipo 1 no es capaz de gestionar correctamente la incertidumbre (al igual que en el Escenario de Estrés), ya que su comportamiento se vuelve muy inestable, con un intervalo de confianza de $\pm7$. En la práctica, implica que, ante condiciones adversas, se vuelve impredecible.
\\
\\
Por el contrario, el Prototipo 2 presenta un comportamiento más estable. No solo consigue mejorar los tiempos medios, reduciéndolos hasta 160 minutos, sino que además lo hace manteniendo una variabilidad muy baja ($\pm2.5$). Por ello, esto indica que hasta en situaciones complicadas como restricciones meteorológicas o fallos operativos, el modelo sigue ofreciendo resultados coherentes.
\section{Evaluación de los Objetivos del TFG}
En esta sección se evalúa si se han cumplido los objetivos definidos al principio de este trabajo correctamente. Para ello, se analizan de forma individual cada uno de los objetivos planteados:
\begin{itemize}
    \item \textbf{Definir y modelar el sistema aeroportuario:} Se han identificado casi en su totalidad todos los elementos esenciales que forman parte del flujo operacional de un aeropuerto. A lo largo del trabajo se ha hablado tanto de la infraestructura como de los procesos clave en la gestión aeroportuaria, y se ha incluido un diagrama físico que recoge los aspectos más importantes para facilitar la compresión del Modelo Empírico.
    \\
    \\
    Como posibles mejoras, se podrían añadir más servicios auxiliares y otros factores que influyen en el funcionamiento de un aeropuerto, como por ejemplo controles adicionales o los sistemas de transporte utilizados para acceder al aeropuerto. No obstante, teniendo en cuenta el alcance de este proyecto, puede considerarse que este objetivo se ha cumplido de manera satisfactoria. 
    \item \textbf{Formular la metodología matemática:} Dentro del Prototipo 1 se ha desarrollado un modelo matemático basado en la \textit{Teoría de Colas}. En este modelo también se han explicado las diferentes fórmulas matemáticas utilizadas, las cuales introducen procesos aleatorios y estocásticos en la simulación.
    \\
    \\
    Como mejora adicional, se podrían haber incorporado más factores aleatorios y más servicios, como por ejemplo pasajeros que llegan tarde al embarque y obligan a bajar su equipaje. o fallo técnicos del avión. Este último caso, en parte, ya se ha considerado mediante la simulación de aterrizajes de emergencia. Aun así, teniendo en cuenta el alcance del proyecto, puede afirmarse que este objetivo se ha cumplido de forma satisfactoria.
    \item \textbf{Implementar y validar un prototipo funcional:} Se ha desarrollado un prototipo que gestiona correctamente la concurrencia entre aeronaves y algunos servicios auxiliares. Cada aeronave cuenta con su propio ciclo, el cual se ejecuta de forma correcta y aleatoria, respetando los límites y horarios establecidos. Por ejemplo, una aeronave no puede aterrizar hasta que haya pasado al menos un minuto desde el aterrizaje de otra, y del mismo modo, no es posible que dos aeronaves ocupen simultáneamente el mismo puesto de estacionamiento.
    \\
    \\
    No obstante, el modelo puede mejorarse mediante la inclusión de nuevos eventos o una gestión más precisa de la concurrencia con los servicios auxiliares. Aun así, en términos generales, puede considerarse que este objetivo también se ha cumplido satisfactoriamente.
    \item \textbf{Analizar e interpretar los resultados obtenidos:} Para cada prototipo se ha intentado realizar una evaluación similar en los escenarios, aunque variado ligeramente el tipo de gráficas utilizadas en cada caso para analizar resultados desde diferentes puntos de vista. Dado que un sistema aeroportuario se generan una gran cantidad de resultados, no ha sido posible abordar todos los aspectos de evaluación, pero sí se han considerado los más representativos, como el tiempo medio ciclo, el número de pasajeros y la carga de trabajo en los distintos turnos de un día.
    \\
    \\
    Aun así, se podrían haber realizado más análisis para obtener conclusiones más completas. De todas formas, los resultados principales han podido representarse de forma adecuada. Por tanto, este objetivo puede considerarse casi cumplido, aunque quedaría margen de mejora en este apartado.
\end{itemize} 

\section{Trabajo Futuro}
Aunque este Trabajo de Fin de Grado ha cumplido con la mayoría de objetivos planteados, es importante señalar que el modelo desarrollado representa solo una parte de la complejidad de un aeropuerto. Esta simplificación del proyecto está condicionada tanto por el tiempo disponible como por el alcance propio de un proyecto de dicha magnitud realizado de forma individual. Aun así, el prototipo final construido constituye una base sólida sobre la que se pueden plantear diversas mejoras y ampliación en trabajos futuros.
\\
\\
En primer lugar, una posible mejora sería la integración de datos reales en tiempo casi real. En lugar de generar las llegadas de forma aleatoria y estadística, el simulador podría alimentarse de programaciones de vuelos reales mediante bases de datos disponibles o APIs de tráfico aéreo. Esto permitiría que la simulación reflejara situaciones reales del día a día del aeropuerto y aumentará su utilidad práctica.
\\
\\
Otra mejora interesante estaría relacionada con la evolución del sistema. Por un lado, la visualización podría mejorarse utilizando motores gráficos como Unity, lo que permitiría mostrar el aeropuerto en un entorno 3D más claro y fácil de interpretar. Por otro lado, se podrían aplicar técnicas de Machine Learning a partir de los datos generados por la simulación. De esta forma, el sistema podría aprender a identificar situaciones de saturación y anticipar posibles problemas operativos con mayor antelación, ayudando así a mejorar la toma de decisiones.
\\
\\
También se podría aumentar el nivel de detalle del ciclo de cada aeronave. En lugar de tratar el servicio en tierra como un solo proceso (estacionamiento), sería posible modelar de forma más específica tareas como la limpieza, el catering o el embarque, es decir, que en el log aparezcan más estados para saber en cada momento en que proceso se encuentra la aeronave. Además, resultaría interesante incluir la simulación del flujo de los pasajeros dentro de la terminal, de forma que se pudiera relacionar el estado de pistas con posibles situaciones de congestión en puertas y zonas comunes.
\\
\\
Por último, otra mejora importante sería el análisis del impacto medioambiental. Teniendo en cuenta que el simulador ya calcula tiempos de espera y de rodaje con los motores en funcionamiento, se podrían añadir estimaciones del consumo de combustible y de las emisiones de C02. De esta manera, el modelo no solo permitiría analizar la eficiencia operativa del aeropuerto, sino que también estudiar posibles medidas para reducir su impacto medioambiental. 
\section{Análisis del Impacto}
\subsection{Impacto General}
El desarrollo de este sistema de simulación no se limita únicamente al ámbito académico, sino que tiene implicaciones en distintos aspectos, tanto a nivel personal como en la mejora de la gestión operativa y económica del aeropuerto.
\begin{itemize}
    \item \textbf{Nivel personal:} A nivel personal, este trabajo me ha ayudado a comprender mejor el funcionamiento de un sistema aeroportuario y la complejidad que conlleva, Además, ha supuesto un reto importante sacar a delante el proyecto y poner en práctica mis conocimientos adquiridos durante la carrera. Durante el desarrollo del TFG he mejorado notablemente mi manejo de Python y he aprendido a utilizar Latex para la redacción de documentos técnicos. Asimismo, he podido profundizar en el uso de simulaciones de eventos y he descubierto que este tipo de trabajos pueden resultar especialmente interesantes y motivadores.
    \item \textbf{Nivel operativo:} En este ámbito, un simulador aeroportuario permite probar escenarios hipotéticos sin asumir riesgos reales. Esto supone un cambio en la gestión del aeropuerto, ya que permite pasar de un modelo reactivo a uno más preventivo, identificando cuellos de botella o situaciones de saturación antes de que ocurran. En un entorno real no es posible realizar este tipo de pruebas como cerrar una pista o duplicar el tráfico aéreo, ya que implicaría un riesgo elevado para la seguridad y la operatividad de un aeropuerto.
    \item \textbf{Nivel económico:} El tiempo es uno de los recursos más valiosos en este tipo de escenarios, ya que cualquier retraso puede suponer un coste económico elevado, tanto en servicios adicionales como en tasas aeroportuarias. Por este motivo, optimizar el tiempo de ciclo de las aeronaves y los tiempos asociados a los distintos procesos permite reducir estos costes y mejorar la eficiencia económica del sistema en general. Esto tiene un impacto positivo tanto para las aerolíneas como para el propio aeropuerto.
\end{itemize}
\subsection{Impacto en los Objetivos de Desarrollo Sostenible}
La simulación de operaciones, flujo de aeronaves y sistemas auxiliares en un aeropuerto es clave para modernizar los aeropuertos. Por ello, el desarrollo de este trabajo contribuye a varios Objetos de Desarrollo Sostenible (ODS), que se exponen a continuación:
\begin{itemize}
    \item \textbf{ODS 8:} \textbf{Trabajo decente y Crecimiento económico.} La simulación desarrollada tiene como objetivo mejorar la eficiencia de las operaciones del aeropuerto. Al reducir los tiempos de espera y los retrasos, se mejora la productividad del aeropuerto, y, por tanto, ayuda a disminuir costes. Además, contar con una herramienta que permite anticipar situaciones de saturación facilita una mejor planificación del trabajo.
    \item \textbf{ODS 9:} \textbf{Industria, Innovación e Infraestructura.} Este proyecto muestra cómo el uso de herramientas de simulación puede ayudar a la mejor gestión de infraestructuras, como un aeropuerto, sin necesidad de realizar cambios físicos. Mediante un modelo software se puede probar varios escenarios y analizar su impacto antes de aplicarlos en la realidad, por lo que es más seguro e innovador.
\end{itemize}



