\chapter{Resultados y Conclusiones}
Una vez implementados y validados ambos prototipos, es necesario analizar los datos obtenidos. En este capítulo se comparan ambos prototipos con el objetivo de evaluar cuál de ellos ofrece un mayor grado de realismo y una representación fiel del flujo de operaciones y sistemas auxiliares. Por último, se presenta una síntesis de los resultados más relevantes del estudio y se proponen algunas ideas para mejorar y ampliar el modelo en trabajos futuros.
\section{Análisis Comparativo de los Prototipos}
El objetivo de esta sección es contrastar los resultados obtenidos en ambos prototipos para identificar las principales diferencias y los cambios más relevantes entre ellos. Mientras que el primer prototipo sirvió como una base inicial, pero muy completa, del modelo, mientras que el segundo se desarrolló con el objetivo de incoporar un mayor nivel de realismo con nuevas lógicas y funcionalidades.
\\
\\
A continuación, se presenta una comparativa mediante tablas en la que se analizan las diferencias entre los escenarios de cada prototipo. En ellas se comparan los aspectos clave de ambos modelos, con el objetivo de identificar en qué puntos difieren y cómo evolucionan entre una versiópn y otra.
\\
\\
Para comenzar se va a realizar una comparativa de los Escenario Base de cada prototipo:
\begin{table}[H]
    \centering
    \renewcommand{\arraystretch}{1.3}
    \begin{tabular}{|l|c|c|} 
        \hline
        Criterio & Prototipo 1 & Prototipo 2\\
        \hline
        Comportamiento General & Lineal y Determinista & Estocástico y Variable\\
        Sensibilidad a la Carga & Baja & Alta\\
        Saturación & Inexistente & Evidente\\
        Tiempo Medio de Ciclo & \textasciitilde81 minutos & \textasciitilde115 minutos\\
        Sesgo & Optimista & Conservador\\
        Variabilidad del IC & Menor ($\pm3$) & Mayor ($\pm5$)\\
        Presencia de Valores Atípicos & Nula & Evidente\\
        Consistencia del Flujo & Alta & Alta\\
        \hline
    \end{tabular}
    \caption{Comparativa Escenario Base}
    \label{tab:Comparativa Escenario Base}
\end{table}

\begin{table}[h!]
    \centering
    \renewcommand{\arraystretch}{1.3}
    \begin{tabular}{|l|c|c|} 
        \hline
        Criterio & Prototipo 1 & Prototipo 2\\
        \hline
        Comportamiento General & Caótico e Inestable & Robusto y Predecible\\
        Sensibilidad a la Carga & Crítica & Alta\\
        Saturación & Colapso & Gestionada\\
        Tiempo Medio de Ciclo & \textasciitilde200 minutos & \textasciitilde169 minutos\\
        Sesgo & Ineficiente & Resiliente\\
        Variabilidad del IC & Muy alta ($\pm7$) & Baja ($\pm3.5$)\\
        Presencia de Valores Atípicos & Descontrolada & Controlada\\
        Consistencia del Flujo & Alta & Alta\\
        \hline
    \end{tabular}
    \caption{Comparativa Escenario de Estrés}
    \label{tab:Comparativa Escenario de Estrés}
\end{table}

\begin{table}[h!]
    \centering
    \renewcommand{\arraystretch}{1.3}
    \begin{tabular}{|l|c|c|} 
        \hline
        Criterio & Prototipo 1 & Prototipo 2\\
        \hline
        Comportamiento General & Inestable y Volátil & Resiliente y Estable\\
        Sensibilidad a la Carga & Muy Alta & Alta\\
        Saturación & Degradación Aleatoria & Degradación Lineal\\
        Tiempo Medio de Ciclo & \textasciitilde171 minutos & \textasciitilde160 minutos\\
        Sesgo & Impreciso & Preciso\\
        Variabilidad del IC & Muy alta ($\pm7$) & Baja ($\pm2.5$)\\
        Presencia de Valores Atípicos & Frecuente & Acotada \\
        Consistencia del Flujo & Alta & Alta\\
        \hline
    \end{tabular}
    \caption{Comparativa Escenario Adverso}
    \label{tab:Comparativa Escenario Adverso}
\end{table}
\section{Conclusiones y Líneas de Trabajo Futuro}
Finalmente, en esta sección se presentan las conclusiones generales del trabajo, así como las posibles líneas de investigación o deasrrollo futuro sobre este tema.



