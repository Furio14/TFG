\chapter{Resultados y Conclusiones}
Una vez implementados y validados ambos prototipos, es necesario analizar los datos obtenidos. En este capítulo se comparan ambos prototipos con el objetivo de evaluar cuál de ellos ofrece un mayor grado de realismo y una representación fiel del flujo de operaciones y sistemas auxiliares. Por último, se presenta una síntesis de los resultados más relevantes del estudio y se proponen algunas ideas para mejorar y ampliar el modelo en trabajos futuros.
\section{Análisis Comparativo de los Prototipos}
El objetivo de esta sección es contrastar los resultados obtenidos en ambos prototipos para identificar las principales diferencias y los cambios más relevantes entre ellos. Mientras que el primer prototipo sirvió como una base inicial, pero muy completa, del modelo, mientras que el segundo se desarrolló con el objetivo de incoporar un mayor nivel de realismo con nuevas lógicas y funcionalidades.
\\
\\
A continuación, se presenta una comparativa mediante tablas en la que se analizan las diferencias entre los escenarios de cada prototipo. En ellas se comparan los aspectos clave de ambos modelos, con el objetivo de identificar en qué puntos difieren y cómo evolucionan entre una versiópn y otra.
\\
\\
Para comenzar se va a realizar una comparativa de los Escenario Base de cada prototipo:
\begin{table}[H]
    \centering
    \renewcommand{\arraystretch}{1.3}
    \begin{tabular}{|l|c|c|} 
        \hline
        Criterio & Prototipo 1 & Prototipo 2\\
        \hline
        Comportamiento General & Lineal y Determinista & Estocástico y Variable\\
        Sensibilidad a la Carga & Baja & Alta\\
        Saturación & Inexistente & Evidente\\
        Tiempo Medio de Ciclo & \textasciitilde81 minutos & \textasciitilde115 minutos\\
        Sesgo & Optimista & Conservador\\
        Variabilidad del IC & Media ($\pm3$) & Alta ($\pm5$)\\
        Presencia de Valores Atípicos & Nula & Evidente\\
        Consistencia del Flujo & Alta & Alta\\
        \hline
    \end{tabular}
    \caption{Comparativa Escenario Base}
    \label{tab:Comparativa Escenario Base}
\end{table}
Tal y como se observa en la Tabla \ref{tab:Comparativa Escenario Base}, el paso del primer al segundo prototipo supone una clara mejora del modelo, ya que se pasa de un enfoque lineal y fijo a uno estacástico. El cambio más notable se observa en el Tiempo Medio de Ciclo, que aumenta de 81 minutos en el primer prototipo a aproximadamente 116 minutos en el segundo prototipo. Es importante destacar que este incremento del 42\% no se debe a un peor funcionamiento, sino a que el primer modelo era demasiado optimista.
\\
\\
Mientras que en el Prototipo 1 la situación era ideal, ya que no había servicios auxiliares, cierres de pista, ni tiempos de espera entre aterrizajes, el Prototipo 2 tiene en cuenta más condiciones realistas del entorno aeroportuario, como la variabilidad en los tiempos de servicio y la llegada de vuelos en estado de emergencia. Además, el análisis de valores atípicos y del intervalo de confianza, que revela una diferencia importante entre ambos modelos. Mientras que el primer modelo presentaba un intervalo estrecho ($\pm3$), lo que no era tan realista, el segundo modelo muestra un intervalo más amplio ($\pm5$), De este modo, el sistema refleja de forma más honesta los problemas operativos y confirma una mayor robustez del modelo.
\begin{table}[h!]
    \centering
    \renewcommand{\arraystretch}{1.3}
    \begin{tabular}{|l|c|c|} 
        \hline
        Criterio & Prototipo 1 & Prototipo 2\\
        \hline
        Comportamiento General & Caótico e Inestable & Robusto y Predecible\\
        Sensibilidad a la Carga & Crítica & Alta\\
        Saturación & Colapso & Gestionada\\
        Tiempo Medio de Ciclo & \textasciitilde200 minutos & \textasciitilde169 minutos\\
        Sesgo & Ineficiente & Resiliente\\
        Variabilidad del IC & Muy alta ($\pm7$) & Media ($\pm3.5$)\\
        Presencia de Valores Atípicos & Descontrolada & Controlada\\
        Consistencia del Flujo & Alta & Alta\\
        \hline
    \end{tabular}
    \caption{Comparativa Escenario de Estrés}
    \label{tab:Comparativa Escenario de Estrés}
\end{table}
\\
Al poner ambos prototipos a prueba en un escenario de estrés, los resultados de la Tabla \ref{tab:Comparativa Escenario de Estrés} son bastante claros. El Prototipo 1, que en situaciones normales parecía funcionar bastante bien, en este escenario se viene abajo, ya que el tiempo medio se dispara a 200 minutos y, además, la incertidumbre aumenta mucho ($\pm7$), lo que indica que el modelo deja de ser estable y se vuelve difícil de predecir.
\\
\\
En cambio, el Prototipo 2 demuestra que está mejor preparado para este tipo de situaciones. Consigue un tiempo medio considerablemente menor (169 minutos) y, además, mantiene la variabilidad más controlada ($\pm3.5$). Por tanto, esto confirma que, cuando el aeropuerto se satura, el segundo modelo es capaz de gestionar mejor las colas y organizar el tráfico.
\begin{table}[h!]
    \centering
    \renewcommand{\arraystretch}{1.3}
    \begin{tabular}{|l|c|c|} 
        \hline
        Criterio & Prototipo 1 & Prototipo 2\\
        \hline
        Comportamiento General & Inestable y Volátil & Resiliente y Estable\\
        Sensibilidad a la Carga & Muy Alta & Alta\\
        Saturación & Degradación Aleatoria & Degradación Lineal\\
        Tiempo Medio de Ciclo & \textasciitilde171 minutos & \textasciitilde160 minutos\\
        Sesgo & Impreciso & Preciso\\
        Variabilidad del IC & Muy alta ($\pm7$) & Baja ($\pm2.5$)\\
        Presencia de Valores Atípicos & Frecuente & Acotada \\
        Consistencia del Flujo & Alta & Alta\\
        \hline
    \end{tabular}
    \caption{Comparativa Escenario Adverso}
    \label{tab:Comparativa Escenario Adverso}
\end{table}
\\
Por último, la Tabla \ref{tab:Comparativa Escenario Adverso} muestra claramente el comportamiento de ambos modelos en condiciones desfavorables. Se observa que el Prototipo 1 no es capaz de gestionar correctamente la incertidumbre (al igual que en el Escenario de Estrés), ya que su comportamiento se vuelve muy inestable, con un intervalo de confianza de $\pm7$. En la práctica, implica que, ante condiciones adversas, se vuelve impredecible.
\\
\\
Por el contrario el Prototipo 2 presenta un comportamiento más estable. No solo consigue mejorar los tiempos medios, reduciéndolos hasta 160 minutos, sino que además lo ahce manteniendo una variabilidad muy baja ($\pm2.5$). Por ello, esto indica que hasta en situaciones complicadas como restricciones meteorológicas o fallos operativos, el modelo sigue ofreciendo resultados coherentes.
\section{Conclusiones y Líneas de Trabajo Futuro}
Finalmente, en esta sección se presentan las conclusiones generales del trabajo, así como las posibles líneas de investigación o deasrrollo futuro sobre este tema.
\subsection{Conclusiones}

\subsection{Líneas de Trabajo Futuro}


